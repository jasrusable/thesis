\documentclass{article}

\usepackage{graphicx}

\title{Positioning using the comparison of images to digital surface models.}
\date{2015}
\author{Jason David Russell}

\begin{document}

\maketitle
\pagenumbering{gobble}

\newpage
\tableofcontents

\newpage
\pagenumbering{arabic}

\section{Background}
\paragraph{}
Personal digital consumer cameras have been with us for quite some time now. The rapid advancement of technology has led to the dramatic decrease in physical size of cameras as well as the rapid increase of image and lens quality, we carry high quality digital cameras with us every day in the form of our smart phone which can store hundreds of high quality images.

\paragraph{}
Digital surface models have become more and more prominent in recent years as the technology used for capturing data such as lidar has become more accessible and feasible for the consumer and prosumer. Digital surface models are typically created using stereo analysis of images, contour analysis or from point clouds. Digital surface models are readily available in South Africa from the NGI.

\paragraph{}
The purpose of this project is to try to resolve the position of a camera by comparing two different images taken from the same point in 2D against a digital surface model to obtain two lines in 2D which can then be intersected to obtain the position of the camera. A secondary purpose, assuming the primary purpose has been achieved, would be to investigate and compare the precisions and accuracies of the various different ways of positioning in outdoor and indoor situations. (such as resection, intersection, GPS fix, etc) An additional purpose would be to determine and explore the applications of resolving position using imagery and digital surface models.

\paragraph{}
Programing will be done in C++ as this language is very popular and fast. C++ has a large community and has support in many popular libraries. OpenCV is one such library which will be made use of for this project. OpenCV is aimed mainly at real-time computer vision and has been under active development since 2000. OpenCV has support for many different programming languages such as C, C++, Pyhton and Java. OpenCV is also capable of running on mobile operating systems Andriod and iOS, so there is a possibility of extending the outcomes of this project into the mobile app market.

\section{Problem Statement}
The coordinates of an unkown point need to be resolved by taking two photographs form the unknown point of objects which are represented in a digital surface model. A program will need to be created in order to match the images of the objects to the ojects mapped on the digital surface model.

\section{Objectives}

\Create a $m^3$ coordinate system out of cardboard in which to perform experiments and tests in.

\paragraph{}
Be able to create a simple digital surface model of two regular objects within the cardboard coordinate system.

\paragraph{}
Be able to recognize and identify the sides of a resolved Rubik's cube using OpenCV. I would do this by labelling each side of the cube using the colour as a key, and then supplying OpenCV with an image of each side and be able to have a program which can identify the side based on the colour.

\paragraph{}
Be able to match an image of two Rubik's cubes

\section{Questions}
\paragraph{}
Can one reliably and accurately determine the location of an unknown position by comparing images of objects taken from that unknown point to a digital surface model containing and reflecting those objects in it? This would enable a person to be able to determine their position in space without the need for GPS, this would also have indoor applications.


\section{Methodology}




\end{document}
